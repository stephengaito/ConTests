% A ConTeXt document [master document: contests.tex]

\startchapter[title=Unit Testing]

Inspired by the Behaviour Driven Development (BDD) paradigm as implemented 
by, for example, Ruby's RSpec testing tool, we use a hierarchical 
structure for our tests. At the top level a particular behaviour is tested 
in a \quote{TestSuite}. Different examples of this behaviour are tested in 
various \quote{TestCases} contained in a \quote{TestSuite}. Finally 
particular \quote{assertions} are listed in one or more \quote{Tests} 
contained in a given \quote{TestCase}. Paralleling the companion Literate 
Programming \ConTeXt\ module, we have three types of tests, 
\quote{ConTests}, \quote{LuaTests} and \quote{CTests} to test collections 
of assertions written in \TeX\ or \ConTeXt\ macros, Lua, or ANSI-C code 
respectively. 

\section[title=TestSuites]

TestSuites provide a high level description of a particular behaviour. A 
TestSuite typically consists of a general description of what behaviour is 
expected and why it is important. It will also often contain some code and 
a high level description of how the code implements the TestSuite's 
overall behaviour.

The TestSuite is implemented as a \ConTeXt\ environment complete with a 
pair of \type{\startTestSuite} and \type{\stopTestSuite} macros. Both of 
these macros make calls, using \type{\directlua}, into corresponding Lua 
code, \type{startTestSuite} and \type{stopTestSuite} methods. These Lua 
methods ensure that the Lua structures used to capture the contents of the 
TestSuite are properly initialized and then stored for later use. 

The \type{\startTestSuite} macro expects one \quote{[} \quote{]} delimited 
argument which should be a short description of the behaviour being 
described by this TestSuite. The contents of this description are used to 
provide a title of a \ConTeXt\ \type{\subsection} macro, as well as being 
passed to the \type{startTestSuite} Lua method. Since the 
\type{startTestSuite} method's description argument is quoted in \type{"} 
characters, you can not use \type{"} in the body of the 
\type{\startTestSuite} argument itself. 

\startMkIVCode
\def\startTestSuite[#1]{%
  \subsection[title=Test Suite: #1]
  \directlua{thirddata.contests.startTestSuite("#1")}
}

\def\stopTestSuite{%
  %\stopsubsection%
  \directlua{thirddata.contests.stopTestSuite()}
}
\stopMkIVCode

The Lua methods used by the TestSuite environment, capture the contents of 
a TestSuite in the \type{thirddata.contests.tests.curSuite} Lua (hash) 
table. The primary entries in the \type{curSuite} table are the 
\type{passed}, \type{cases}, and \type{desc} keys. The \type{passed} key 
holds a boolean which asserts whether or not the current TestSuite has 
passed (or failed). The \type{cases} key holds an array of TestCases as 
each is defined as part of the current TestSuite. Finally the \type{desc} 
key holds a Lua string which is the user's short description of this 
TestSuite. 

\startLuaCode
local function initSuite()
  tests.stage     = ''
  local curSuite  = {}
  curSuite.passed = true
  curSuite.cases  = {}
  return curSuite
end

function contests.startTestSuite(aDesc)
  tests.curSuite      = initSuite()
  tests.curSuite.desc = aDesc
end

function contests.stopTestSuite()
  tInsert(tests.suites, tests.curSuite)
  tests.curSuite = initSuite()
end
\stopLuaCode

\section[title=TestCases]

TestCases provide a high level description of an example of a behaviour. A 
TestCase typically consists of a general description of the example and 
why the example is important to a given behaviour. A TestCase will contain 
one or more Tests, which can be ConTests, LuaTests or CTests. A TestCase 
might also contain some code and a high level description of how the code 
helps structure the example. 

Again, the TestCase is implemented as a \ConTeXt\ environment with 
\type{\startTestCase} and \type{\stopTestCase} macros. For very good 
logistical reasons, a designer might want to sketch a collection of 
TestCases, but not yet want to have them actually tested. For this reason 
we have included an alternate end of environment macro, 
\type{\skipTestCase}, which can be used to tell the Lua code to finish the 
current TestCase but that it should \emph{not} be tested. These three 
\ConTeXt\ macros use \type{\directlua} to call corresponding Lua methods, 
\type{startTestCase}, \type{stopTestCase} and \type{skipTestCase}. 

The \type{\startTestCase} macro expects one \quote{[} \quote{]} delimited 
argument which should be a short description of the example being tested 
by this TestCase. The contents of this description is used to provide the 
\quote{title} of a \quote{TestCase} \ConTeXt\ \quote{text rule} 
environment. This description is also passed to the \type{startTestCase} 
Lua method. Since the \type{startTestCase} method's description argument 
is quoted in \type{"} characters, you can not use \type{"} in the body of 
the \type{\startTestCase} argument itself. 

\startMkIVCode
\def\startTestCase[#1]{%
  \starttextrule{Test case}
  \noindent {\tfa #1} \godown[2ex]
  \directlua{thirddata.contests.startTestCase("#1")}
}

\def\stopTestCase{%
  \directlua{thirddata.contests.stopTestCase()}
  \stoptextrule%
}

\def\skipTestCase{%
  \directlua{thirddata.contests.skipTestCase()}
  \stoptextrule%
}
\stopMkIVCode

The Lua methods used by the TestCase environment, make sure any previous 
TestCases are properly saved in the current TestSuite. They then 
initialize a new TestCase in the \type{curSuite.curCase} (hash) table. 
Particularly important keys in the \type{curCase} table are the 
\type{passed}, \type{desc}, \type{fileName}, \type{startLine}, and 
\type{lastLine}. The \type{passed} key holds a boolean which asserts 
whether or not the current TestCase has passed (or failed). The 
\type{desc} key holds the short description provided by the user. The 
\type{fileName}, \type{startLine} and \type{lastLine} keys hold 
information about where the current TestCase is located in the source 
document. This information is used by the failure reporting system, 
described below, to help the user identify where in their source code, any 
TestCases are failing to perform as expected. 

The \type{stopTestCase} and \type{skipTestCase} methods, both ensure that 
the current TestCase is appended to the TestSuite's \type{case} entry for 
any future use.

To be extensible we provide the ability to add test runner Lua methods to 
the \type{thirddata.contests.testRunners} Lua (hash) table. Any Lua method 
found in the \type{thirddata.contests.testRunner} \emph{hash} will be run 
by the \type{stopTestCase} method at the end of every test case. These 
test runner Lua methods must accept two arguments, the \type{curSuite} and 
the \type{curCase}, which are the Lua table containing the current test 
suite and test case respectively. 

This \ConTeXt\ module defines two test runners, \type{runCurMkIVTestCase} 
and \type{runLuaTestCase}, to run the ConTest and LuaTest tests defined by 
a given test case. These two test runners are defined below in their 
respective sections as part of the overall ConTest or LuaTest unit test 
definitions. 

\startLuaCode
function contests.startTestCase(aDesc)
  local suite       = tests.curSuite
  suite.curCase     = {}
  local curCase     = suite.curCase
  curCase.passed    = true
  curCase.desc      = aDesc
  curCase.fileName  = status.filename
  curCase.startLine = status.linenumber
end

function contests.stopTestCase()
  local curSuite   = tests.curSuite
  local curCase    = curSuite.curCase
  curCase.lastLine = status.linenumber
  for runnerName, runner in pairs(contests.testRunners) do
    if type(runner) == 'function' then
      runner(curSuite, curCase)
    end
  end
  tInsert(curSuite.cases, curCase)
end

function contests.skipTestCase()
  local curSuite   = tests.curSuite
  local curCase    = curSuite.curCase
  curCase.lastLine = status.linenumber
  tInsert(curSuite.cases, curCase)
  if curCase.mkiv and 0 < #curCase.mkiv then
    local caseStats = mkivStats.cases
    caseStats.skipped = caseStats.skipped + 1
  end
  if curCase.lua and 0 < #curCase.lua then
    local caseStats = luaStats.cases
    caseStats.skipped = caseStats.skipped + 1
  end
  curCase.cTests = { }
  tex.print('{\\magenta SKIPPED}')
end
\stopLuaCode

\section[title=Reporting failures]

Having defined a collection of TestSuites, TestCases and ConTests, 
LuaTests or CTests, a user needs to know which TestCases and Tests have 
either passed or failed. In particular, it is important to systematically 
list all assertion failures to ensure they can be understood and 
corrected. 

Failure reporting occurs at two distinct levels. At the lowest level, each 
assertion logs a failure. At the highest, document level, the program 
author needs to be able to make use of various \ConTeXt\ macros to provide 
human readable reports either summarizing or detailing all failures. 

The Lua \type{logFailure} method is used to create a \quote{failure 
object}, (a Lua hash table) which contains a collection of keys describing 
a given failure. The resulting table is used in various places to both 
report the failure directly or to store the failure information for later 
reporting as a consolidated list of failures. 

\startLuaCode
local function logFailure(reason, suiteDesc, caseDesc,
                          testMsg, errMsg, fileInfo)
  local failure = {}
  failure.reason    = reason
  failure.suiteDesc = suiteDesc
  failure.caseDesc  = caseDesc
  failure.testMsg   = testMsg
  failure.errMsg    = errMsg
  failure.fileInfo  = fileInfo
  return failure
end
\stopLuaCode

The \type{reportFailure} method uses \type{tex.print} to ensure a failure 
report is typeset by \TeX\ and visible to the reader of the document. The 
\type{fullReport} argument is a boolean which determines whether or not 
the TestSuite and TestCase descriptions associated with the failure report 
are to be reported to the user in a given context. When reporting failures 
at the end of a TestCase, this context is not important and only provides 
a distraction. When reported in the consolidated list of all failures, the 
TestSuite and TestCase descriptions provide valuable context for the 
reader to determine what when wrong and possibly why. 

\startLuaCode
local function reportFailure(aFailure, fullReport)
  tex.print("\\noindent{\\red "..aFailure.reason.."}:\\\\")
  if fullReport then
    tex.print(aFailure.suiteDesc.."\\\\")
    tex.print(aFailure.caseDesc.."\\\\")
  end
  if aFailure.testMsg and 0 < #aFailure.testMsg then
    tex.print(aFailure.testMsg.."\\\\")
  end
  tex.cprint(12, aFailure.errMsg)
  tex.print("\\\\"..aFailure.fileInfo)
end
\stopLuaCode

The \type{\reportFailures} macro uses the \type{reportFailures} and 
\type{reportFailure} Lua methods to provide a consolidated itemized list 
of all failures detected in the document's unit tests. 

\startMkIVCode
\def\reportFailures{%
  \directlua{thirddata.contests.reportFailures()}
}
\stopMkIVCode

\startLuaCode
function contests.reportFailures()
  if 0 < #tests.failures then
    tex.print("\\startitemize ")
    for i, aFailure in ipairs(tests.failures) do
      tex.print("\\item ")
      reportFailure(aFailure, true)
    end
    tex.print("\\stopitemize ")
  else
    tex.print("{\\green All test cases PASSED}")
  end
end
\stopLuaCode

The \type{\reportMkIVStats} and \type{\reportLuaStats} macros use the 
\type{reportStats} Lua method to build a \ConTeXt\ table which can be 
typeset for the reader to see how many TestCases and Assertions have been 
attempted, passed or failed. These statistics provide a quick summary of 
how well tested a given code project is. 

\startMkIVCode
\def\reportMkIVStats{%
  \directlua{thirddata.contests.reportStats('mkiv')}
}

\def\reportLuaStats{%
  \directlua{thirddata.contests.reportStats('lua')}
}
\stopMkIVCode

The \type{reportStats} Lua method uses the \ConTeXt\ \type{\placetable} 
and \quote{tabulate} environment macros to build a simple table of summary 
statistics associated with a given type of test. 

\startLuaCode
function contests.reportStats(statsType)
  local stats = tests.stats[statsType]
  local rows = { 'cases', 'assertions' }
  local cols =
    { 'attempted', 'passed', 'failed', 'skipped' }
  local colCol = { '', '\\green', '\\red', '\\magenta' }
  tex.print("\\placetable[force,none]{}{%")
  tex.print("\\starttabulate[|r|c|c|c|c|]\\HL\\NC")
  for j, col in ipairs(cols) do
    tex.print("\\NC "..colCol[j]..' '..col)
  end
  tex.print("\\NR\\HL")
  for i, row in ipairs(rows) do
    tex.print("\\NC "..row)
    for j, col in ipairs(cols) do
      tex.print("\\NC "..colCol[j]..' '..tostring(stats[row][col]))
    end
    tex.print("\\NR")
  end
  tex.print("\\HL\\stoptabulate")
  tex.print("}")
end
\stopLuaCode

\stopchapter