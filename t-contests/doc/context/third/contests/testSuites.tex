% A ConTeXt document [master document: contests.tex]

\startchapter[title=Unit Test Suites]

Note that you can not use \type{"} in the \type{\startTestSuite} or 
\type{\startTestCase} arguments. 

\startMkIVCode

\def\startTestSuite[#1]{%
  \startsubsection[title=Test Suite: #1]
  \directlua{thirddata.contests.startTestSuite("#1")}
}

\def\stopTestSuite{%
  \stopsubsection%
  \directlua{thirddata.contests.collectTestSuite()}
}

\def\startTestCase[#1]{%
  \starttextrule{Test case}
  \noindent {\tfa #1} \godown[2ex]
  \directlua{thirddata.contests.startTestCase("#1")}
}

\def\stopTestCase{%
  \directlua{thirddata.contests.collectTestCase()}
  \stoptextrule%
}

\def\reportFailures{%
  \directlua{thirddata.contests.reportFailures()}
}

\stopMkIVCode

\startLuaCode

local function initSuite()
  local curSuite = {}
  curSuite.passed = true
  curSuite.cases  = {}
  return curSuite
end

function contests.startTestSuite(aDesc)
  tests.curSuite      = initSuite()
  tests.curSuite.desc = aDesc
end

function contests.collectTestSuite()
  table_insert(tests.suites, tests.curSuite)
  tests.curSuite = initSuite()
end

function contests.startTestCase(aDesc)
  local suite       = tests.curSuite
  suite.curCase     = {}
  local curCase     = suite.curCase
  curCase.passed    = true
  curCase.desc      = aDesc
  curCase.fileName  = status.filename
  curCase.startLine = status.linenumber
end

function contests.collectTestCase()
  local curSuite   = tests.curSuite
  local curCase    = curSuite.curCase
  curCase.lastLine = status.linenumber
  contests.runCurLuaTestCase(curSuite, curCase)
  table_insert(curSuite.cases, curCase)
  curSuite.curCase = {}
end

function contests.reportStats(statsType)
  local stats = tests.stats[statsType]
  local rows = { 'cases', 'assertions' }
  local cols =
    { 'attempted', 'passed', 'failed' }
  local colCol = { '', '\\green', '\\red' }
  stats.assertions.unexpected =
    stats.assertions.failed - stats.assertions.expected
  tex.print("\\placetable[force,none]{}{%")
  tex.print("\\starttabulate[|r|c|c|c|]\\HL\\NC")
  for j, col in ipairs(cols) do
    tex.print("\\NC "..colCol[j]..' '..col)
  end
  tex.print("\\NR\\HL")
  for i, row in ipairs(rows) do
    tex.print("\\NC "..row)
    for j, col in ipairs(cols) do
      tex.print("\\NC "..colCol[j]..' '..tostring(stats[row][col]))
    end
    tex.print("\\NR")
  end
  tex.print("\\HL\\stoptabulate")
  tex.print("}")
end

local function logFailure(reason, suiteDesc, caseDesc,
                          testMsg, errMsg, fileInfo)
  local failure = {}
  failure.reason    = reason
  failure.suiteDesc = suiteDesc
  failure.caseDesc  = caseDesc
  failure.testMsg   = testMsg
  failure.errMsg    = errMsg
  failure.fileInfo  = fileInfo
  return failure
end

local function reportFailure(aFailure, fullReport)
  tex.print("\\noindent{\\red "..aFailure.reason.."}:\\\\")
  if fullReport then
    tex.print(aFailure.suiteDesc.."\\\\")
    tex.print(aFailure.caseDesc.."\\\\")
  end
  if aFailure.testMsg and 0 < #aFailure.testMsg then
    tex.print(aFailure.testMsg)
  end
  tex.cprint(12, aFailure.errMsg)
  tex.print("\\\\"..aFailure.fileInfo)
end

function contests.reportFailures()
  if 0 < #tests.failures then
    tex.print("\\startitemize ")
    for i, aFailure in ipairs(tests.failures) do
      tex.print("\\item ")
      reportFailure(aFailure, true)
    end
    tex.print("\\stopitemize ")
  else
    tex.print("{\\green All test cases PASSED}")
  end
end

\stopLuaCode

\stopchapter