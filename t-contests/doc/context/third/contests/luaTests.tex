% A ConTeXt document [master document: contests.tex]

\startchapter[title=Lua Unit Testing]

We make extensive use of ideas in the design of the LunaTest Lua unit 
testing framework which is in turn inspired by the LUnit assertion 
interface. We tailor these ideas for use \emph{inside} \ConTeXt. 

\section[title=Test cases]

% temporarily include this in the main document
% until the actual module files stabilize
\definetyping[LuaTest]
\setuptyping[LuaTest][option=lua]

\startMkIVCode

\definetyping[LuaTest]
\setuptyping[LuaTest][option=lua]

\let\oldStopLuaTest=\stopLuaTest
\def\stopLuaTest{%
  \oldStopLuaTest%
  \directlua{thirddata.contests.addLuaTest('_typing_')}
}

\stopMkIVCode

\startLuaCode

function contests.addLuaTest(bufferName)
  local bufferContents = buffers.getcontent(bufferName):gsub("\13", "\n")
  local suite = tests.curSuite
  local case  = suite.curCase
  case.lua    = case.lua or {}
  table_insert(case.lua, bufferContents)
end

function contests.runCurLuaTestCase()
  local suite    = tests.curSuite
  local case     = suite.curCase
  local luaChunk = table_concat(case.lua, '\n')
  if not luaChunk:match('^%s*$') then
    -- consider using PCall here
    local luaFunc, errMessage = load(luaChunk)
    if luaFunc then
      local result = luaFunc()
    end
  end
end

\stopLuaCode

\section[title=Assertions]

We start by defining a helper function to capture the repetitive code in 
one place. Every assertion will contain a condition, a message and some 
reason for failing the assertion. The \type{reportLuaAssertion} function 
captures these three values and builds an assertion result \emph{if} the 
condition is false. If the condition is true, the 
\type{reportLuaAssertion} function quietly does nothing but update the 
test statistics. 

\startLuaCode
function reportLuaAssertion(theCondition, aMessage, theReason)
  -- we do not need to do anything unless theCondition if false!
  if not theCondition then
    local test     = { }
    test.message   = aMessage
    test.reason    = theReason
    test.condition = theCondition
    local info     = debug.getinfo(2,'l')
    test.line      = info.currentline
    error(test, 0) -- throw an error to be captured by an error_handler
  end
end
\stopLuaCode

Since all of the assertions will use the standard lua \type{string.format} 
function to format the various reasons using information specific to each 
assertion, we make \type{string.format} local with a shorter name. 

\startLuaCode
local fmt = string.format
\stopLuaCode

Having defined both the \type{reportLuaAssertion} and the \type{fmt} 
functions, we now work through each assertion in turn. For each assertion, 
we begin by defining the lua code which implements the assertion, and then 
the test cases which verify the assertion is working correctly. Typically 
there will be two test cases for each assertion, we need to test both the 
positive and negative conditions. 

\startTestSuite[assert_throwsError]

In order to test if a lua function throws an error, we need to wrap the 
actual function call (together with its arguments) in a lua \type{pcall}. 
The return value of this \type{pcall} is then checked to see if an error 
was thrown. 

\startLuaCode
function contests.assert_throwsError(aFunction, aMessage, ...)
  local ok, err = pcall(aFunction, ...)
  return reportLuaAssertion(
    ok,
    aMessage,
    fmt("Expected %s to throw an error.", TS(aFunction))
  )
end
\stopLuaCode

There are two cases we need to test.

\startTestCase[should \succeed\ if error thrown]

In this first test case, we want to make sure if the tested function 
throws an error then the assertion succeeds. 

\startLuaTest
  assert_throwsError(function()
    error('this should throw an error!')
  end)
\stopLuaTest
\stopTestCase

\startTestCase[should \fail\ if no error thrown]

In this second test case, we want to make sure the assertion \emph{fails} 
if the tested function \emph{does not} throw an error.

\startLuaTest
  assert_throwsError(assert_throwsError(function() end))
\stopLuaTest
\stopTestCase

\stopTestSuite


\startTestSuite[assert_throwsNoError]

\startLuaCode
function contests.assert_throwsNoError(aFunction, aMessage, ...)
  local ok, err = pcall(aFunction, ...)
  return reportLuaAssertion(
    not ok,
    aMessage,
    fmt("Expected %s not to throw an error (%s).", TS(aFunction), TS(err))
  )
end
\stopLuaCode

\stopTestSuite

\startTestSuite[assert_fail]

\startLuaCode
function contests.assert_fail(aMessage)
  return reportLuaAssertion(
    false,
    aMessage,
    "(Failed)"
  )
end
\stopLuaCode

\startTestCase[should fail]

\startLuaTest
  something
\stopLuaTest

\stopTestCase
\stopTestSuite

\startTestSuite[assert_isBoolean]

\startLuaCode
function contests.assert_isBoolean(anObj, aMessage)
  return reportLuaAssertion(
    type(anObj) == 'boolean',
    aMessage,
    fmt("Expected %s to be a boolean.", TS(anObj))
  )
end
\stopLuaCode

\stopTestSuite

\startTestSuite[assert_isNotBoolean]

\startLuaCode
function contests.assert_isNotBoolean(anObj, aMessage)
  return reportLuaAssertion(
    type(anObj) ~= 'boolean',
    aMessage,
    fmt("Expected %s to not be a boolean.", TS(anObj))
  )
end
\stopLuaCode

\stopTestSuite

\startTestSuite[assert_isTrue]

\startLuaCode
function contests.assert_isTrue(aBoolean, aMessage)
  return reportLuaAssertion(
    aBoolean,
    aMessage,
    fmt("Expected true, got %s.", TS(aBoolean))
  )
end
\stopLuaCode

\startTestCase[should succeed]

\startLuaTest
  something else
\stopLuaTest

\stopTestCase
\stopTestSuite

\startTestSuite[assert_isFalse]

\startLuaCode
function contests.assert_isFalse(aBoolean, aMessage)
  return reportLuaAssertion(
    not aBoolean,
    aMessage,
    fmt("Expected false, got %s.", TS(aBoolean))
  )
end
\stopLuaCode

\stopTestSuite

\startTestSuite[assert_isNil]

\startLuaCode
function contests.assert_isNil(anObj, aMessage)
  return reportLuaAssertion(
    anObj == nil,
    aMessage,
    fmt("Expected nil, got %s.", TS(anObj))
  )
end
\stopLuaCode

\stopTestSuite

\startTestSuite[assert_isNotNil]

\startLuaCode
function contests.assert_isNotNil(anObj, aMessage)
  return reportLuaAssertion(
    anObj ~= nil,
    aMessage,
    fmt("Expected non-nil, got %s.", TS(anObj))
  )
end
\stopLuaCode

\stopTestSuite

\startTestSuite[assert_isEqual]

\startLuaCode
function contests.assert_isEqual(anObj, expected, aMessage)
  return reportLuaAssertion(
    anObj == expected,
    aMessage,
    fmt("Expected %s to equal %s.", TS(expected), TS(anObj))
  )
end
\stopLuaCode

\stopTestSuite

\startTestSuite[assert_isEqualWithIn]

\startLuaCode
function contests.assert_isEqualWithIn(anObj, expected, tolerance, aMessage)
  return reportLuaAssertion(
    true, --????,
    aMessage,
    fmt("Expected %s to equal %s with tolerance %s.",
      TS(anObj), TS(expected), TS(tolerance))
  )
end
\stopLuaCode

\stopTestSuite

\startTestSuite[ssert_isNotEqual]

\startLuaCode
function contests.assert_isNotEqual(anObj, expected, aMessage)
  return reportLuaAssertion(
    anObj ~= expected,
    aMessage,
    fmt("Expected %s to not equal %s.", TS(expected), TS(anObj))
  )
end
\stopLuaCode

\stopTestSuite

\startTestSuite[assert_isNotEqualWithIn]
\startLuaCode
function contests.assert_isNotEqualWithIn(anObj, expected, tolerance, aMessage)
  return reportLuaAssertion(
    true, --????,
    aMessage,
    fmt("Expected %s to not equal %s with tolerance %s.",
      TS(anObj), TS(expected), TS(tolerance))
  )
end
\stopLuaCode

\stopTestSuite

\startTestSuite[assert_isNumber]

\startLuaCode
function contests.assert_isNumber(anObj, aMessage)
  return reportLuaAssertion(
    type(anObj) == 'number',
    aMessage,
    fmt("Expected %s to be a number.", TS(anObj))
  )
end
\stopLuaCode

\stopTestSuite

\startTestSuite[assert_isGT]

\startLuaCode
function contests.assert_isGT(objA, objB, aMessage)
  return reportLuaAssertion(
    objA > objB,
    aMessage,
    fmt("Expected %s > %s.", TS(objA), TS(objB))
  )
end
\stopLuaCode

\stopTestSuite

\startTestSuite[assert_isGTE]

\startLuaCode
function contests.assert_isGTE(objA, objB, aMessage)
  return reportLuaAssertion(
    objA >= objB,
    aMessage,
    fmt("Expected %s >= %s.", TS(objA), TS(objB))
  )
end
\stopLuaCode

\stopTestSuite

\startTestSuite[assert_isLT]

\startLuaCode
function contests.assert_isLT(objA, objB, aMessage)
  return reportLuaAssertion(
    objA < objB,
    aMessage,
    fmt("Expected %s < %s.", TS(objA), TS(objB))
  )
end
\stopLuaCode

\stopTestSuite

\startTestSuite[assert_isLTE]

\startLuaCode
function contests.assert_isLTE(objA, objB, aMessage)
  return reportLuaAssertion(
    objA <= objB,
    aMessage,
    fmt("Expected %s <= %s.", TS(objA), TS(objB))
  )
end
\stopLuaCode


\stopTestSuite

\startTestSuite[assert_isNotNumber]

\startLuaCode
function contests.assert_isNotNumber(anObj, aMessage)
  return reportLuaAssertion(
    type(anObj) ~= 'number',
    aMessage,
    fmt("Expected %s to be a number.", TS(anObj))
  )
end
\stopLuaCode

\stopTestSuite

\startTestSuite[assert_isString]

\startLuaCode
function contests.assert_isString(anObj, aMessage)
  return reportLuaAssertion(
    type(anObj) == 'string',
    aMessage,
    fmt("Expected [%s] to be a string.", TS(anObj))
  )
end
\stopLuaCode

\stopTestSuite

\startTestSuite[assert_matches]

\startLuaCode
function contests.assert_matches(anObj, aPattern, aMessage)
  return reportLuaAssertion(
    anObj:matches(aPattern),
    aMessage,
    ftm("Expected [%s] to match [%s].", TS(anObj), TS(aPattern))
  )
end
\stopLuaCode

\stopTestSuite

\startTestSuite[assert_doesNotMatch]

\startLuaCode
function contests.assert_doesNotMatch(anObj, aPattern, aMessage)
  return reportLuaAssertion(
    not anObj:matches(aPattern),
    aMessage,
    fmt("Expected [%s] to not match [%s].", TS(anObj), TS(aPattern))
  )
end
\stopLuaCode

\stopTestSuite

\startTestSuite[assert_length]

\startLuaCode
function contests.assert_length(anObj, aLength, aMessage)
  return reportLuaAssertion(
    #anObj == aLength,
    aMessage,
    fmt("Expected %s to have length %s.", TS(anObj), TS(aMessage))
  )
end
\stopLuaCode

\stopTestSuite

\startTestSuite[assert_isNotLength]

\startLuaCode
function contests.assert_isNotLength(anObj, aLength, aMessage)
  return reportLuaAssertion(
    #anObj ~= aLength,
    aMessage,
    fmt("Expected %s to not have length %s.", TS(anObj), TS(aMessage))
  )
end
\stopLuaCode

\stopTestSuite

\startTestSuite[assert_isNotString]

\startLuaCode
function contests.assert_isNotString(anObj, aMessage)
  return reportLuaAssertion(
    type(anObj) ~= 'string',
    aMessage,
    fmt("Expected [%s] to not be a string.", TS(anObj))
  )
end
\stopLuaCode

\stopTestSuite

\startTestSuite[assert_isTable]

\startLuaCode
function contests.assert_isTable(anObj, aMessage)
  return reportLuaAssertion(
    type(anObj) == 'table',
    aMessage,
    fmt("Expected %s to be a table.", TS(anObj))
  )
end
\stopLuaCode

\stopTestSuite

\startTestSuite[assert_hasKey]

\startLuaCode
function contests.assert_hasKey(anObj, aKey, aMessage)
  return reportLuaAssertion(
    anObj[aKey] ~= nil,
    aMessage,
    fmt("Expected %s to have the key %s.", TS(anObj), TS(aKey))
  )
end
\stopLuaCode

\stopTestSuite

\startTestSuite[assert_doesNotHaveKey]

\startLuaCode
function contests.assert_doesNotHaveKey(anObj, aKey, aMessage)
  return reportLuaAssertion(
    anObj[aKey] == nil,
    aMessage,
    fmt("Expected %s to not have the key %s.", TS(anObj), TS(aKey))
  )
end
\stopLuaCode

\stopTestSuite

\startTestSuite[assert_isNotTable]

\startLuaCode
function contests.assert_isNotTable(anObj, aMessage)
  return reportLuaAssertion(
    type(anObj) ~= 'table',
    aMessage,
    fmt("Expected %s to not be a table.", TS(anObj))
  )
end
\stopLuaCode

\stopTestSuite

\startTestSuite[assert_isFunction]

\startLuaCode
function contests.assert_isFunction(anObj, aMessage)
  return reportLuaAssertion(
    type(anObj) == 'function',
    aMessage,
    fmt("Expected %s to be a function.", TS(anObj))
  )
end
\stopLuaCode

\stopTestSuite

\startTestSuite[assert_isNotFunction]

\startLuaCode
function contests.assert_isNotFunction(anObj, aMessage)
  return reportLuaAssertion(
    type(anObj) ~= 'function',
    aMessage,
    fmt("Expected %s to not be a function.", TS(anObj))
  )
end
\stopLuaCode

\stopTestSuite

\startTestSuite[assert_isUserData]

\startLuaCode
function contests.assert_isUserData(anObj, aMessage)
  return reportLuaAssertion(
    type(anObj) == 'userdata',
    aMessage,
    fmt("Expected %s to be user data.", TS(anObj))
  )
end
\stopLuaCode

\stopTestSuite

\startTestSuite[assert_isNotUserData]

\startLuaCode
function contests.assert_isNotUserData(anObj, aMessage)
  return reportLuaAssertion(
    type(anObj) ~= 'userdata',
    aMessage,
    fmt("Expected %s to not be user data.", TS(anObj))
  )
end
\stopLuaCode

\stopTestSuite

\startTestSuite[assert_hasMetaTable]

\startLuaCode
function contests.assert_hasMetaTable(anObj, aMessage)
  return reportLuaAssertion(
    getMetaTable(anObj) ~= nil,
    aMessage,
    fmt("Expected %s to have a meta table.", TS(anObj))
  )
end
\stopLuaCode

\stopTestSuite

\startTestSuite[assert_metaTableEquals]

\startLuaCode
function contests.assert_metaTableEqual(anObj, aMetaTable, aMessage)
  return reportLuaAssertion(
    getMetaTable(anObj) == aMetaTable,
    aMessage,
    fmt("Expected %s to have the meta table %s.", TS(anObj), TS(aMetaTable))
  )
end
\stopLuaCode

\stopTestSuite

\startTestSuite[assert_metaTableNotEqual]

\startLuaCode
function contests.assert_metaTableNotEqual(anObj, aMetaTable, aMessage)
  return reportLuaAssertion(
    getMetaTable(anObj) ~= aMetaTable,
    aMessage,
    fmt("Expected %s to not have the meta table %s.", TS(anObj), TS(aMetaTable))
  )
end
\stopLuaCode

\stopTestSuite

\startTestSuite[assert_doesNotHaveMetaTable]

\startLuaCode
function contests.assert_hasMetaTable(anObj, aMessage)
  return reportLuaAssertion(
    getMetaTable(anObj) == nil,
    aMessage,
    fmt("Expected %s to not have a meta table.", TS(anObj))
  )
end
\stopLuaCode

\stopTestSuite

\startTestSuite[assert_isThread]

\startLuaCode
function contests.assert_isThread(anObj, aMessage)
  return reportLuaAssertion(
    type(anObj) == 'thread',
    aMessage,
    fmt("Expected %s to be a thread.", TS(anObj))
  )
end
\stopLuaCode

\stopTestSuite

\startTestSuite[assert_isNotThread]

\startLuaCode
function contests.assert_isNotThread(anObj, aMessage)
  return reportLuaAssertion(
    type(anObj) ~= 'thread',
    aMessage,
    fmt("Expected %s to not be a thread.", TS(anObj))
  )
end
\stopLuaCode

\stopTestSuite

\stopchapter