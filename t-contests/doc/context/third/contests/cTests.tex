% A ConTeXt document [master document: contests.tex]

\chapter[title=C Unit Testing]

We will eventually include a collection of C unit tests.

\startMkIVCode
\definetyping[CTest]
\setuptyping[CTest][option=cpp]

\let\oldStopCTest=\stopCTest
\def\stopCTest{%
  \oldStopCTest%
  \directlua{thirddata.contests.addCTest('_typing_')}
}

\def\createCTestFile#1#2#3{%
  \relax
}
\stopMkIVCode

\startLuaCode
function contests.addCTest(bufferName)
  local bufferContents = buffers.getcontent(bufferName):gsub("\13", "\n")
  local suite = tests.curSuite
  local case  = suite.curCase
  case.ansiC  = case.ansiC or {}
  tInsert(case.ansiC, bufferContents)
end

function contests.collectCTest()

end
\stopLuaCode

\startCHeader
#ifndef FALSE
#define FALSE 0
#endif

#ifndef TRUE
#define TRUE 1
#endif

#ifndef NULL
#define NULL 0
#endif

typedef struct TestCase_struc TestCase;

typedef void (*TestCaseFunc)(TestCase*);

struct TestCase_struc {
  char*         desc;
  TestCaseFunc  testCaseFunc;
  int           failed;
  const char*   message;
  const char*   reason;
  jmp_buf      *jumpBuf;
};

extern void TestCaseRun(TestCase *tc);

extern void ReportCAssert(TestCase *tc,
                          int theCondition,
                          const char* aMessage,
                          const char* theReson,
                          const char* fileName,
                          int lineNum);

extern char* sprintf2Int(const char* fmt,
                         int intA, int intB);
extern char* sprintf2Ptr(const char* fmt,
                         void* ptrA, void* ptrB);
extern char* sprintf2Str(const char* fmt,
                         const char* strA,
                         const char* strB);
extern char* sprintf2Dbl(const char* fmt,
                         double dblA, double dblB);
\stopCHeader

\startCCode
void TestCaseRun(TestCase *tc) {
  tc->attempted = 0;
  tc->passed    = 0;
  tc->failed    = FALSE;
  jmp_buf jumpBuf;
  tc->jumpBuf = &jumpBuf;
  (tc->testCaseFunc)(tc);
  tc->jumpBuf = 0;
}

void ReportCAssert(TestCase *tc,
                   int theCondition,
                   const char* aMessage,
                   const char* theReason,
                   const char* fileName,
                   int lineNum) {
  tc->attempted++;
  if ! theCondition {
    tc->message  = aMessage;
    tc->reason   = theReason;
    tc->fileName = fileName;
    tc->lineNum  = lineNum;
    tc->failed   = TRUE;
    // Now do a long jump back to the TestCaseFunc
    //
    if (tc->jumpBuf) longjump(*(tc->jumpBuf), 0);
  }
  tc->passed++;
}

#define getBuffer(bufferSize) (char*)calloc(1, bufferSize)

char* sprintf2Int(const char* fmt,
                  int intA, int intB) {
  size_t bufferSize = strlen(fmt) + 20*2;
  char* buffer = getBuffer(bufferSize);
  snprintf(buffer, bufferSize, fmt, intA, intB);
  return buffer;
}

char* sprintf2Ptr(const char* fmt,
                  void* ptrA, void* ptrB) {
  size_t bufferSize = strlen(fmt) + 20*2;
  char* buffer = getBuffer(bufferSize);
  snprintf(buffer, bufferSize, fmt, ptrA, ptrB);
  return buffer;

char* sprintf2Str(const char* fmt,
                  const char* strA,
                  const char* strB) {
  size_t bufferSize = strlen(fmt)
    + strlen(strA) + strlen(strB) + 10;
  char* buffer = getBuffer(bufferSize);
  snprintf(buffer, bufferSize, fmt, strA, strB);
  return buffer;
}

char* sprintf2Dbl(const char* fmt,
                  double dblA, double dblB) {
  size_t bufferSize = strlen(fmt) + 20*2;
  char* buffer = getBuffer(bufferSize);
  snprintf(buffer, bufferSize, fmt, dblA, dblB);
  return buffer;
}

\stopCCode

\section[title=Assertions]

\startTestSuite[assertFail]

\startCHeader
#define AssertFailMsg(tc, aMessage) \
  ReportCAssert(                    \
    (tc),                           \
    FALSE,                          \
    (aMessage),                     \
    "Failed",                       \
    __FILE__,                       \
    __LINE__)

#define AssertFail(tc) \
  AssertFailMsg(tc, "")
\stopCHeader

\stopTestSuite

\startTestSuite[assertSucceed]

\startCHeader
#define AssertSucceedMsg(tc, aMessage) \
  ReportCAssert(                       \
    (tc),                              \
    TRUE,                              \
    (aMessage),                        \
    "Succeeded",                       \
    __FILE__,                          \
    __LINE__)

#define AssertSucceed(tc) \
  AssertSucceedMsg(tc, "")
\stopCHeader

\startTestSuite[assertIntTrue]

\startCHeader
#define AssertIntTrueMsg(tc, anInt, aMessage) \
  ReportCAssert(                              \
    (tc),                                     \
    (anInt),                                  \
    (aMessage),                               \
    sprintf2Int(                              \
      "Expected %d to be TRUE.",              \
      (anInt),                                \
      NULL                                    \
    ),                                        \
    __FILE__,                                 \
    __LINE__)

#define AssertIntTrue(tc, anInt) \
  AssertIntTrueMsg(tc, anInt, "")
\stopCHeader

\stopTestSuite

\startTestSuite[assertIntFalse]

\startCHeader
#define AssertIntFalseMsg(tc, anInt, aMessage) \
  ReportCAssert(                               \
    (tc),                                      \
    !(anInt),                                  \
    (aMessage),                                \
    sprintf2Int(                               \
      "Expected %d to be FALSE.",              \
      (anInt),                                 \
      NULL                                     \
    ),                                         \
    __FILE__,                                  \
    __LINE__)

#define AssertIntFalse(tc, anInt) \
  AssertIntFalseMsg(tc, anInt, "")
\stopCHeader

\stopTestSuite

\startTestSuite[assertIntEquals]

\startCHeader
#define AssertIntEqualsMsg(tc, intA, intB, aMessage) \
  ReportCAssert(                                     \
    (tc),                                            \
    (intA) == (intB),                                \
    (aMessage),                                      \
    sprintf2Int(                                     \
      "Expected %d to be equal to %d.",              \
      (intA),                                        \
      (intB)                                         \
    ),                                               \
    __FILE__,                                        \
    __LINE__)

#define AssertIntEquals(tc, intA, intB) \
  AssertIntEqualsMsg(tc, intA, intB, "")
\stopCHeader

\stopTestSuite

\startTestSuite[assertIntNotEquals]

\startCHeader
#define AssertIntNotEqualsMsg(tc, intA, intB, aMessage) \
  ReportCAssert(                                        \
    (tc),                                               \
    (intA) != (intB),                                   \
    (aMessage),                                         \
    sprintf2Int(                                        \
      "Expected %d not to be equal to %d.",             \
      (intA),                                           \
      (intB)                                            \
    ),                                                  \
    __FILE__,                                           \
    __LINE__)

#define AssertIntNotEquals(tc, intA, intB) \
  AssertIntNotEqualsMsg(tc, intA, intB, "")
\stopCHeader

\stopTestSuite

\startTestSuite[assertPrtNull]

\startCHeader
#define AssertPtrNullMsg(tc, aPtr, aMessage) \
  ReportCAssert(                             \
    (tc),                                    \
    (aPtr) == (NULL),                        \
    (aMessage),                              \
    sprintf2Ptr(                             \
      "Expected %p to be NULL.",             \
      (aPtr),                                \
      NULL                                   \
    ),                                       \
    __FILE__,                                \
    __LINE__)

#define AssertPtrNull(tc, aPtr) \
  AssertPtrNullMsg(tc, aPtr, "")
\stopCHeader

\stopTestSuite

\startTestSuite[assertPtrNotNull]

\startCHeader
#define AssertPtrNotNullMsg(tc, aPtr, aMessage) \
  ReportCAssert(                                \
    (tc),                                       \
    (aPtr) != (NULL),                           \
    (aMessage),                                 \
    sprintf2Pt(                                 \
      "Expected %p not to be NULL.",            \
      (aPtr),                                   \
      NULL                                      \
    ),                                          \
    __FILE__,                                   \
    __LINE__)

#define AssertPtrNotNull(tc, aPtr) \
  AssertPtrNotNullMsg(tc, aPtr, "")
\stopCHeader

\stopTestSuite

\startTestSuite[assertPtrEquals]

\startCHeader
#define AssertPtrEqualsMsg(tc, ptrA, ptrB, aMessage) \
  ReportCAssert(                                     \
    (tc),                                            \
    (ptrA) == (ptrB),                                \
    (aMessage),                                      \
    sprintf2Ptr(                                     \
      "Expected %p to be equal to %p.",              \
      (ptrA),                                        \
      (ptrB)                                         \
    ),                                               \
    __FILE__,                                        \
    __LINE__)

#define AssertPtrEquals(tc, ptrA, ptrB) \
  AssertPtrEqualsMsg(tc, ptrA, ptrB, "")
\stopCHeader

\stopTestSuite

\startTestSuite[assertPtrNotEquals]

\startCHeader
#define AssertPtrNotEqualsMsg(tc, ptrA, ptrB, aMessage) \
  ReportCAssert(                                        \
    (tc),                                               \
    (ptrA) != (ptrB),                                   \
    (aMessage),                                         \
    sprintf2Ptr(                                        \
      "Expected %p not to be equal to %p.",             \
      (ptrA),                                           \
      (ptrB)                                            \
    ),                                                  \
    __FILE__,                                           \
    __LINE__)

#define AssertPtrNotEquals(tc, ptrA, ptrB) \
  AssertPtrNotEqualsMsg(tc, ptrA, ptrB, "")
\stopCHeader

\stopTestSuite

\startTestSuite[assertStrEmpty]

\startCHeader
#define AssertStrEmptyMsg(tc, aStr, aMessage) \
  ReportCAssert(                              \
    (tc),                                     \
    *(aStr) == 0,                             \
    (aMessage),                               \
    sprintf2Str(                              \
      "Expected [%s] to be empty.",           \
      (aStr),                                 \
      NULL                                    \
    ),                                        \
    __FILE__,                                 \
    __LINE__)

#define AssertStrEmpty(tc, aStr) \
  AssertStrEmptyMsg(tc, aStr, "")
\stopCHeader

\stopTestSuite

\startTestSuite[assertStrNotEmpty]

\startCHeader
#define AssertStrNotEmptyMsg(tc, aStr, aMessage) \
  ReportCAssert(                                 \
    (tc),                                        \
    *(aStr) != 0,                                \
    (aMessage),                                  \
    sprintf2Str(                                 \
      "Expected [%s] not to be empty.",          \
      (aStr),                                    \
      NULL                                       \
    ),                                           \
    __FILE__,                                    \
    __LINE__)

#define AssertStrNotEmpty(tc, aStr) \
  AssertStrNotEmptyMsg(tc, aStr, "")
\stopCHeader

\stopTestSuite

\startTestSuite[assertStrEquals]

\startCHeader
#define AssertStrEqualsMsg(tc, strA, strB, aMessage) \
  ReportCAssert(                                     \
    (tc),                                            \
    strcmp((strA), (strB)) == 0,                     \
    (aMessage),                                      \
    sprintf2Str(                                     \
      "Expected [%s] to be equal to [%s].",          \
      (strA),                                        \
      (strB)                                         \
    ),                                               \
    __FILE__,                                        \
    __LINE__)

#define AssertStrEquals(tc, strA, strB) \
  AssertStrEqualsMsg(tc, strA, strB, "")
\stopCHeader

\stopTestSuite

\startTestSuite[assertStrNotEquals]

\startCHeader
#define AssertStrNotEqualsMsg(tc, strA, strB, aMessage) \
  ReportCAssert(                                        \
    (tc),                                               \
    strcmp((strA), (strB)) != 0,                        \
    (aMessage),                                         \
    sprintf2Str(                                        \
      "Expected [%s] not to be equal to [%s].",         \
      (strA),                                           \
      (strB)                                            \
    ),                                                  \
    __FILE__,                                           \
    __LINE__)

#define AssertStrNotEquals(tc, strA, strB) \
  AssertStrNotEqualsMsg(tc, strA, strB, "")
\stopCHeader

\stopTestSuite

\startTestSuite[assertDblEquals]

\startCHeader
#define AssertDblEqualsMsg(tc, dblA, dblB, tol, aMessage)    \
  ReportCAssert(                                             \
    (tc),                                                    \
    fabs((dblA) - (dblB)) < (tol),                           \
    (aMessage),                                              \
    sprintf2Dbl(                                             \
      "Expected %e to be equal to %e with tolerance of %e.", \
      (dblA),                                                \
      (dblB)                                                 \
    ),                                                       \
    __FILE__,                                                \
    __LINE__)

#define AssertDblEquals(tc, dblA, dblB) \
  AssertDblEqualsMsg(tc, dblA, dblB, "")
\stopCHeader

\stopTestSuite

\startTestSuite[assertDblNotEqals]

\startCHeader
#define AssertDblNotEqualsMsg(tc, dblA, dblB, tol, aMessage)     \
  ReportCAssert(                                                 \
    (tc),                                                        \
    (tol) <= fabs((dblA) - (dblB)),                              \
    (aMessage),                                                  \
    sprintf2Dbl(                                                 \
      "Expected %e not to be equal to %e with tolerance of %e.", \
      (dblA),                                                    \
      (dblB)                                                     \
    ),                                                           \
    __FILE__,                                                    \
    __LINE__)

#define AssertDblNotEquals(tc, dblA, dblB) \
  AssertDblNotEqualsMsg(tc, dblA, dblB, "")
\stopCHeader

\stopTestSuite
